\documentclass[a4paper,12pt]{article}%twoside
\usepackage{amsfonts, amsmath, amssymb, bm, color, fancyhdr, graphicx, lastpage, listings, natbib, relsize, titling}
\usepackage{hyperref}
\hypersetup{pdftex,colorlinks=true,allcolors=blue}
\usepackage{hypcap}
\usepackage[left=2.5cm,right=2cm]{geometry}
\definecolor{listinggray}{gray}{0.9}
\definecolor{lbcolor}{rgb}{0.9,0.9,0.9}
\usepackage{caption}
\DeclareCaptionFont{black}{\color{black}}
\DeclareCaptionFormat{listing}{\colorbox{\color{listinggray}}{\parbox{0.988\textwidth}{#1#2#3}}}
\captionsetup[lstlisting]{format=listing,labelfont=white,textfont=white}
\lstset{
backgroundcolor=\color{lbcolor},
    tabsize=4,    
%   rulecolor=,
    language=[GNU]C++,
        basicstyle=\scriptsize,
        breakatwhitespace,
        upquote=true,
        aboveskip={1.5\baselineskip},
        columns=fixed,
        showstringspaces=false,
        extendedchars=false,
        breaklines=true,
        prebreak = \raisebox{0ex}[0ex][0ex]{\ensuremath{\hookleftarrow}},
        frame=single,
        numbers=left,
        showtabs=false,
        showspaces=false,
        showstringspaces=false,
        identifierstyle=\ttfamily,
        keywordstyle=\color[rgb]{0,0,1},
        commentstyle=\color[rgb]{0.026,0.112,0.095},
        stringstyle=\color[rgb]{0.627,0.126,0.941},
        numberstyle=\color[rgb]{0.205, 0.142, 0.73},
%      \lstdefinestyle{C++}{language=C++,style=numbers}’.
}
\lstset{
  backgroundcolor=\color{lbcolor},
  breaklines=true,
  %literate={\_}{}{0\discretionary{\_}{}{\_}}%
  tabsize=4,
  language=C++,
  captionpos=t,
  tabsize=3,
  frame=lines,
  numbers=left,
  numberstyle=\tiny,
  numbersep=5pt,
  showstringspaces=false,
  basicstyle=\footnotesize,
%  identifierstyle=\color{magenta},
  keywordstyle=\color[rgb]{0,0,1},
  commentstyle=\color[rgb]{0,0.6,0},
  stringstyle=\color{red}
  }
 
\newcommand\CC{C\nolinebreak[4]\hspace{-.05em}\raisebox{.4ex}{\relsize{-3}{\textbf{++}}}}

%\fancyhead[LE,RO]{\thepage}
\pagestyle{fancy}
%\fancyhead{}% Remove all header contents
\textheight 24cm
\voffset -0.25cm
\footskip 3.5cm
\setlength{\parindent}{0pt}

\makeatletter
%\renewcommand{\headrulewidth}{0pt}% Remove header rule
\renewcommand{\thefootnote}{\roman{footnote}}	

\renewcommand\section{\@startsection{section}{2}{\z@}%
                                     {-3.25ex\@plus -1ex \@minus -.2ex}%
                                     {1.5ex \@plus .2ex}%
                                     {\normalfont\large\bfseries}}% from \large
\renewcommand\subsection{\@startsection{subsection}{2}{\z@}%
                                     {-3.25ex\@plus -1ex \@minus -.2ex}%
                                     {1.5ex \@plus .2ex}%
                                     {\normalfont\bfseries}}
\renewcommand\subsubsection{\@startsection{subsection}{2}{\z@}%
                                     {-3.25ex\@plus -1ex \@minus -.2ex}%
                                     {1.5ex \@plus .2ex}%
                                     {\normalfont\bfseries\itshape}}% from \large                                                                        

% we use \prefix@<level> only if it is defined
\renewcommand{\@seccntformat}[1]{%
  \ifcsname prefix@#1\endcsname
    \csname prefix@#1\endcsname
  \else
    \csname the#1\endcsname\quad
  \fi}
% define \prefix@section
%\newcommand\prefix@section{Section \thesection: }
\newcommand\prefix@section{}

\renewcommand{\@seccntformat}[1]{%
  \ifcsname prefix@#1\endcsname
    \csname prefix@#1\endcsname
  \else
    \csname the#1\endcsname\quad
  \fi}
% define \prefix@section
%\newcommand\prefix@section{Section \thesection: }
\newcommand\prefix@subsection{}

\renewcommand{\@seccntformat}[1]{%
  \ifcsname prefix@#1\endcsname
    \csname prefix@#1\endcsname
  \else
    \csname the#1\endcsname\quad
  \fi}
% define \prefix@section
%\newcommand\prefix@section{Section \thesection: }
\newcommand\prefix@subsubsection{}

\renewcommand{\baselinestretch}{1.15} 

\pretitle{\noindent\Large\bfseries}
\posttitle{\\}
\preauthor{\itshape}
\postauthor{\\}
\pretitle{\begin{center}\Large\bfseries}
\posttitle{\end{center}\\}
\preauthor{\begin{center}\normalsize}
\postauthor{\end{center}\\}
\predate{\begin{center}\itshape}
\postdate{\end{center}\\}
\setlength{\droptitle}{-1.25 in}
\lhead{PiE $\CC$ Final Assignment Report}
\rhead{Duc Nguyen \quad \thepage\ / \pageref{LastPage}}
\title{PiE $\CC$ Final Assignment Report}
%\date{\vspace{-5ex}}
\author{Name: Duc Nguyen, \\student number: s1630512\thanks{Email address for 
correspondence: nguyenmanhduc@student.utwente.nl or ngmaduc@gmail.com.    The codes were composed by and tested with NetBeans IDE 8.0.2 in Mac OSX Yosemite 10.10.5.}}

\begin{document}
\clearpage\maketitle
\thispagestyle{empty}
\pagestyle{fancy}
\newpage
%\maketitle
\section{Exercise 1}
\subsection{Question 1.1}
Write a $\CC$ program to compute the first $N$ prime numbers, where $N$ is given by the user. Use dynamic arrays to store the primes and use this information in the mod test. 
\subsubsection{Answer.}
Three functions were built for this question including $\lstinline{bruteForce, modTest}$ and 
$\lstinline{print_primes}$ for searching prime numbers and the last one for printing out the results in console. Passing by reference were chosen to avoid unnecessary copy of variables.
\begin{lstlisting}
std::vector <unsigned long int> bruteForce (int &n);
std::vector <unsigned long int> modTest (int &n);
void print_primes (int &n, const std::vector <unsigned long int>& primes);
\end{lstlisting}
The method chosen to create dynamic arrays to store the primes is $\lstinline{std::vector}$. The range $[0, 2147483647]$ of  $\lstinline{unsigned long int}$ fits to the scope of the question. The idea of $\lstinline{bruteForce}$,\linebreak $\lstinline{modTestDiv}$ are shown in the following code snippets.
\begin{lstlisting}[title = $\lstinline{bruteForce}$]
       primes.push_back(2);
        unsigned long int c;//need to be the type of primes for mode test
        int count = 1;
        for (int count = 1; count < n; ){//counter from 1; "2" included before
            for (c = 2; c < num; c++){
                if (num % c == 0) {//mod test from 2 to n
                    break;	
                }    
            }
            if (c == num) {//to this point means no divisor up to n, Prime!
                primes.push_back(num);//push to result vector of Prime
                count++;//increase counter   
            }
            num++;  
        }
\end{lstlisting}

\begin{lstlisting}[title = $\lstinline{modTestDiv}$]
        primes.push_back(2);  
        for (int count = 1; count < n; ) {//counter from 1; "2" included before 
            bool isPrime = true;
            for (int i = 0; i < primes.size(); i++){
                if (num % primes[i] ==0) {//non-primes  are products of primes
                  isPrime = false;
                  break;
                }    
            }
            if (isPrime == true) {
                primes.push_back(num);
                count++;//increase counter
            }
            num++;
        }
\end{lstlisting}
\subsection{Question 1.2}
Write to the screen a list of the first 10000 primes in the format given; where $p(n)$ is the
$n^{th}$ prime number. Report only the last five lines. Comment on the behaviour of the ratio $n * ln(p(n))/p(n)$ as n gets large.
\subsubsection{Answer.}
The $\lstinline{void print_ratio (int &n, const std::vector <unsigned long int>& primes)}$ and  prime number search functions together generate the required ratio. As $n$ gets large, the ratio tends to converge to 1. Until $10^5$-th prime number, the ratio is approximately 1.103.\\\\
The last five lines and function $\lstinline{print_ratio}$ are listed below:
\begin{lstlisting}
9996     :       104707  :       1.10348856177824
9997     :       104711  :       1.10356044403989
9998     :       104717  :       1.10361306655082
9999     :       104723  :       1.10366568381267
10000    :       104729  :       1.10371829582629
\end{lstlisting}

\begin{lstlisting}[title = $\lstinline{print_ratio}$]
void print_ratio (int &n, const std::vector <unsigned long int>& primes){
    std::cout << "n\t:\t p(n)\t:\t n*ln( p(n) )/p(n) " << std::endl;
    for (int i = primes.size; i < primes.size(); i++) {
        std::cout << i+1 << "\t:\t" << primes[i] <<"\t:\t" 
            <<std::fixed << std::setprecision(14) << 
                double((i+1)*log(primes[i])/primes[i])<< std::endl;
        }//set precision used for increase decimal displayed
}
\end{lstlisting}
\subsection{Question 1.3}
Based on question 2, give an estimate of the $10^6$-th prime number.
\subsubsection{Answer.}
We use 1.1 for the value of the ratio with $n=10^6$:
$$10^6 * ln(p(10^6))/p(10^6) \approx 1.1 $$
Using Wolfram Alpha to solve this equation, the estimate of the $10^6$-th prime number is:
$$p(10^6) \approx 15022800$$
\subsection{Question 1.4}
Instead of writing to the screen, write to a file (on disk) a list containing just the prime
numbers. Print eight numbers per line, such that all numbers have the same space.
\subsubsection{Answer.}
The $\lstinline{primes_to_file}$ function for writing to a file (on disk) with eight numbers per line is shown below: 
\begin{lstlisting}[title = $\lstinline{primes_to_file}$]
void primes_to_file (int &n, const std::vector <unsigned long int>& primes,
        const std::string& fileName){
        if(n >= 1){
            std::ofstream fileOut;
            fileOut.open(fileName);
            for (int i = 0; i < primes.size(); i++) {
                fileOut << primes[i] << "\t";
                if ((i+1) % 8 == 0) {//Print eight numbers per line
                    fileOut<<std::endl;
                }	
            }
        }
        else{
            std::cerr << "Invalid Input" << std::endl;
        }
}
\end{lstlisting}
\subsection{Question 1.5}
Time your code for $N = 10^3; 10^4; 10^5$ and $10^6$. Make a log-log plot of run-time against $N$ for both codes. What can we say from the log-log plot? Do this analysis for brute force and suggested speed up and comment on the results.
\subsubsection{Answer.}
Figure \eqref{fig1} shows the log-log plot of run-time against $N$. It can be seen that within the range of $N$ the linear relationship between the logarithmic values of run-time and that of $N$. This also mean that the run-time is propotional to $N$ to the power of the slope of the straight line of the log-log graph. Hence, the search function $\lstinline{modTestDiv}$ is faster than $\lstinline{bruteForce}$. However, the $\lstinline{bruteForce}$ can be simply modified to $\lstinline{bruteForceM}$ which results in faster run-time. The only modification is instead of performing mod test from 2 to $n$ for each number $n$, we only do mod test from 2 to square root of $n$, because a non-prime number number $n$ always has a divisor less than or equal square root of $n$. Otherwise, the number $n$ is a prime number. Measurement of run-time was recored by using $\lstinline{std::chrono::time_point<std::chrono::system_clock>}$. The run-time values have been saved to a file for further processing. All the codes were time-measured with the function $\lstinline{rtime_to_file}$. The code snippets for  $\lstinline{rtime_to_file}$ and $\lstinline{bruteForceM}$  are shown below:
\begin{figure}[h]
\centering
\includegraphics[width=0.8\linewidth, height=0.5\linewidth]{rt-vs-n.png}
\caption{log-log plot of run-time against $N$}
\label{fig1}
\end{figure}
\begin{lstlisting}[title = $\lstinline{rtime_to_file}$]
void rtime_to_file (const std::string& fileName, 
    std::function<std::vector <unsigned long int>(double &n)> &f){    
 	...
    for (int i = 0; i < v.size(); i++) {
        std::chrono::time_point<std::chrono::system_clock> start, end;
        start = std::chrono::system_clock::now();
        std::vector<unsigned long int> primes = f(v[i]);//only measure this function
        end = std::chrono::system_clock::now();
        std::chrono::duration<double> elapsed_seconds = end-start;
 	...    
    }
}
\end{lstlisting}
\begin{lstlisting}[title = $\lstinline{bruteForceM}$]
for (int count = 2; count <= n; ){//counter from 1; "2" included before
            for (c = 2; c <= sqrt(num); c++){//non prime has divisor lt sqrt
                ...
            }
            if (c > sqrt(num)) {//to this point no divisor up to n, Prime!
                ...
            }
            num++;  
        }
\end{lstlisting}
\subsection{Question 1.6}
More efficient ways of computing prime numbers exist. Find and implement one and report
the analysis of part 5 for this algorithm. Comment on the results.
\subsubsection{Answer.}
Sieve of Eratosthenes was used for illustrating an efficient way to search for prime numbers. The log-log plot with run-time of Sieve of Eratosthenes against $N$ is shown in Figure \eqref{fig2}.
\begin{figure}[h!]
\centering
\includegraphics[width=0.8\linewidth, height=0.5\linewidth]{rt-vs-n-eratos.png}
\caption{log-log plot of run-time against $N$}
\label{fig2}
\end{figure}\\
The idea is just to mark all the numbers in the range of interest as prime then update the marks as non-prime if a number is the multiplication of the previous primes. It illustrates the fundamental property of prime numbers that any non-prime number can be represented by a multiplication of prime numbers. The code snippet implementing Sieve of Eratosthenes is shown below. The Sieve of Eratosthenes requires the largest number to be found as an input. We can use the ratio from Question 1.3 or Rosser's theorem which states that $p(n) < n * log(n*log(n))$.\\\\
For searching large prime numbers, the linear relationship in the log-log plot will not be held. Further study on complexity of algorithm for prime search function is beyond the scope of this report. However, a simple test can show that for only 7 digits prime number, high complexity methods like $\lstinline{bruteForce}$ already take the order of days. Therefore, they should not be used for finding large prime numbers. 
\begin{lstlisting}[title = $\lstinline{Eratosthenes}$]
	unsigned long int max = n * std::log(n*std::log(n)); //Rosser's theorem
    for (unsigned long int p=2; p < max; p++){ // for all elements in array
        if (primes.size() > n-1){//keep track first n prime only, vector count from 0
            break;
        }
        else if(isPrime[p] == true){ // it is not multiple of any other prime
            primes.push_back(p);
        }
        // mark all multiples of prime selected above as non primes
        int c=2;
        int mul = p * c;
        while(mul <= max){
            isPrime[mul] = false;
            c++;
            mul = p*c;
        }        
    }
\end{lstlisting}
\section{Exercise 2}
\subsection{Question 2.1}
Read a string input from the terminal (which is assumed to be in RPN). Interpret the string
correctly and output the result to the screen.
Your Reverse Polish Notation calculator should be able to do add, subtract, multiply and
divide integers.
\subsubsection{Answer.}
Four functions have been built for an RPN evaluator for integers: $\lstinline{RPN}$ is used for performing arithmethic operations while $\lstinline{parserPostFix}$ and $\lstinline{getFSMCol}$ are used to deal with multi-digit and negative integers. We also need $\lstinline{bool isoperator(char arg)}$ to recognize operators while scanning through a RPN expression. To check for a digit we can use a $\CC$ built-in function $\lstinline{bool isdigit}$. The code use the symbol $\sim$ to designate negative numbers.
\begin{lstlisting}
double RPN (const std::vector<std::string>& expr);
std::vector<std::string> parserPostFix(std::string& postfix);
int getFSMCol(char& currentChar);
bool isoperator(char arg);
\end{lstlisting}
The main idea is to parse $\lstinline{std::string& postfix}$ to a $\lstinline{std::vector<std::string>}$ then perform RPN evaluator. The algorithm for RPN evaluator used in this code is described in programming reference \cite[]{roberts2013}.
Parsing function $\lstinline{parserPostFix}$ was done by using a Finite State Machine (FSM) (listed below) to keep track of keyboard strokes.  It will recognize the consecutive keyboard strokes to combine to a multi-digit or negavtive number then store to a vector element as a string. RPN evaluator will then convert those strings to integer with $\lstinline{c_str}$ before implement its algorithm. 
\begin{lstlisting}
std::array <std::array<int,5>, 5> stateTable= 
{{{0, INTEGER,  NEGATIVE, OPERATOR, SPACE},
  {INTEGER,  INTEGER,  RESTART,  RESTART,   RESTART},
  {NEGATIVE, INTEGER,  RESTART,  RESTART,   RESTART},
  {OPERATOR,  RESTART, RESTART,  RESTART,   RESTART},
  {SPACE,     RESTART, RESTART,  RESTART,   RESTART}}};
\end{lstlisting}
\begin{lstlisting}
if(currentState == RESTART){
            if(currentToken != " "){
                tokens.push_back(currentToken);//push to new cell
            }
            currentToken = "";
        }
        else{
            //recording multi digit and negative until next RESTART
            currentToken += currentChar;
            ++i;
        }
\end{lstlisting}
The first column of FSM can be undestood as the first keyboard stroke and the second column can be understood as the second keyboard stroke. The highlight here is whenever the state is RESTART, a new element in vector is ready to be assigned a value while if the state is not RESTART during several keyboard strokes like in the case of multi-digit or negavtive numbers, the code will continue to combine characters to a string in the current vector element.
\subsection{Question 2.2}
Extend your code such that it reads the input line-by-line from a file. Each newline marks
the end of each calculation. Write out the result of each line of the calculation and your
program should abort when it detects an `end of file' condition.
\subsubsection{Answer.}
Function $\lstinline{inputToPostfix}$ with the powerful $\lstinline{std::getline(fileIn,line)}$ check will produce a \linebreak  $\lstinline{std::vector<std::string>postfix}$ with each element corresponding to an RPN expression per line. Running a simple loop through this vector and do the same as Question 2.1 will write results on the console.
\begin{lstlisting}[title = $\lstinline{inputToPostfix}$]
std::vector<std::string> inputToPostfix (const std::string& fileName){
    std::vector<std::string> postfix;
    std::ifstream fileIn(fileName);
    std::string line;
    while (std::getline(fileIn , line)){
        postfix.push_back(line);
    }
    return postfix;
}
\end{lstlisting}
\subsection{Question 2.3}
Add in error checking to confirm the entered string can be interpreted as RPN string.
\subsubsection{Answer.}
The structure of $\lstinline{try, catch, throw}$ serves as error checking. Errors were defined from $\lstinline{RPN}$ function. Everytime an error happens, there will be a throw of $\lstinline{std::runtime_error}$ which then will be catched in main fucntion using $\lstinline{e.what()}$ . The list of errors are described in the code snippets below:
\begin{lstlisting}	
	if (s.size() < 2){//each operator require 2 operands
    		throw std::runtime_error("Not enough operands"); 
     }
\end{lstlisting}
\begin{lstlisting}	
   if (secondOperand == 0){
  		throw std::runtime_error("Attempt to divide by zero.");
   }
\end{lstlisting}
\begin{lstlisting}	
   if (s.size() != 1){//last element to be pop out after arithmethic operation is the result
        throw std::runtime_error("Invalid Input");
    }                       
\end{lstlisting}
\subsection{Question 2.4}
Add in support for the power operator using the symbol $\^{ }$
\subsubsection{Answer.}
RPN evaluator does not need an order to be defined for operations. Thus to add in support for the power operator using the symbol $\^{ }$  we only need to add a case in arithmetic operation $\lstinline{switch}$ and a check of the symbol $\^{ }$  in $\lstinline{bool isoperator(char arg)}$.

\begin{lstlisting}[title = $\lstinline{isoperator}$]
bool isoperator(char arg){
    if(arg == '*' || arg == '/' || arg == '+' || arg == '-' || || arg == '^'){ 
        return(1);
    }
    else{ 
        return(0);
    }
}                      
\end{lstlisting}

\begin{lstlisting}	
   case ('^'):
   		ans = std::pow(firstOperand, secondOperand);
    		break;                    
\end{lstlisting}
\subsection{Question 2.5}
Extend your code so that it can read input strings in normal brackets notation. It should
understand the BODMAS rules.
\subsubsection{Answer.}
The solution consists of two steps: the first step is to convert infix to postfix notation and the second step is to perfrom RPN evaluator. The latter step is already done upto question 2.4. For the first step, fucntion $\lstinline{infixToPostfix}$ were built. We also need function $\lstinline{precendence}$ to determine an order for operations the before we can implement Shunting-yard \cite[]{kushwaha2014}, which was also invented by Edsger Dijkstra. This code can handle both reading infix notation from file and from user input. The fucntions needed for infix notation calculator for integers are listed below:
\begin{lstlisting}
std::vector<std::string> inputToInfix (const std::string& fileName);
std::string infixToPostfix(std::string& infix); 
bool isoperator(char arg);
int precendence(char& arg);
double RPN (const std::vector<std::string>& expr);
std::vector<std::string> parserPostFix(std::string& postfix);
int getFSMCol(char& currentChar); 
\end{lstlisting}
Function $\lstinline{precendence}$ is simply assign values to operator characters following BODMAS rule using $\lstinline{switch}$. Note that brackets were not assigned a value because it will be processed seperately in $\lstinline{infixToPostfix}$. Function $\lstinline{infixToPostfix}$ serves as another error checking espically for unexpected and imbalance breackets. Those two above-mentioned functions are listed in the code snippets below:
\begin{lstlisting}[title = $\lstinline{precendence}$]
int precendence(char& arg){//Add weight to the operator, high priority high value
    int weight = 0;
    switch(arg){
    case '^':
        weight = 3;
        break;
    case '*':
    case '/':
        weight = 2;
        break;
    case '+':
    case '-':
        weight = 1;
        break;
    }
    return(weight);
}
\end{lstlisting}

\begin{lstlisting}
	while((!s.empty())&&(precendence(infix[i]) <= precendence(s.top()))){
	      postfix+= s.top();//add to RPN expression
              s.pop();
     }
\end{lstlisting}
\begin{lstlisting}
	else if(infix[i] == '('){
            if (expectingOperator == true){
                throw std::runtime_error("Operator Missing");
            }
            s.push(infix[i]);
        }
        else if(infix[i] == ')'){
            if (expectingOperator == false){
                throw std::runtime_error("Operand Missing");
            }
            while((!s.empty()) && (s.top() != '(')){
	        postfix += s.top();		
		s.pop();
            }
            if(s.empty()){
                throw std::runtime_error("Parentheses Mismatch");
            }
            s.pop();
            expectingOperator = true;
        }
        else{//no operator                        
            throw std::runtime_error("Only integers and +,-,*,/,^ are allowed");
        }
\end{lstlisting}
\begin{lstlisting}
	while(!s.empty()){
            if (s.top() == '('){
                throw std::runtime_error("Parentheses Mismatch");
            }
            postfix+= s.top();
            s.pop();//pop everything from stack to complete RPN expression
        }
\end{lstlisting}
        
\section{Exercise 3}
\subsection{Question 3.1}
Use any algorithm to compute the shortest distance between every set of cities and write this information to disk with the route as a list of cities.
\subsubsection{Answer.}
The question asks for the shortest distance between every set of cities thus Floyd-Warshall’s algorithm which is well-known for solving to solve the All-Pairs-Shortest-Path problem was chosen. This report only covers key points of the algorithm as well as tries to explain how to implement it it $\CC$. The details of Floyd-Warshall’s algorithm used for the code in this exercise is described in Graph Theory reference. \cite[]{ray2013}\\

The input file given is very suitable for the algorithm because it is already in the form of adjacency matrix (distance matrix) which gives all information about the cities and how they are connected. They are also known as nodes and edges' length in graph theory. The distance matrix $d$ is stored by 2-D vector $\lstinline{std::vector<std::vector<int>> d}$. To reconstruct the shortest path, we also need a node sequence matrix $s$ which is also stored by a 2-D $\lstinline{std::vector<std::vector<int>> s}$. All the elements of node sequence matrix $s$ are initially zero which means that the initial shortest path is the direct connection from citi $i$ to city $j$ for each pair $i,j$. The size of both vectors are the square of nodes. The functions $\lstinline{lines_count}$ and $\lstinline{input2vector}$ were used to obtain the number of nodes and create the distance matrix, respectively. We are now ready to implement Floyd-Warshall’s algorithm which is done by the function $\lstinline{WFI}$. 

\begin{lstlisting}
int lines_count (const std::string& fileName);
std::vector<std::vector<int>> input2vector (const std::string& fileName);
void WFI(int &nodes, std::vector<std::vector<int>>& d, 
        std::vector<std::vector<int>>& s);
int main()
{  
	...    
    std::vector<std::vector<int>> s(nodes,std::vector<int>(nodes, 0)); 
    ...
}
\end{lstlisting}

The code snippet for the implementation of  $\lstinline{lines_count}$ and $\lstinline{input2vector}$ are shown below:
\begin{lstlisting}[title = $\lstinline{lines_count}$]
std::ifstream fileIn(fileName);
    int n =0;
    std::string line;
    while (std::getline(fileIn , line)){
        n++;//increase n after each line
    }
\end{lstlisting}

\begin{lstlisting}[title = $\lstinline{input2vector}$]
while (std::getline(fileIn , line)){
        std::vector<int> lineData;
        std::istringstream  lineStream(line);
        int value;
        // Read an integer at a time from the line
        while(lineStream >> value){
            // Add the integers from a line to a 1D vector
            lineData.push_back(value);
        }
        // When all the integers have been read add the 1D array
        // into a 2D array (as one line in the 2D array)
        d.push_back(lineData);
    }
\end{lstlisting}

Floyd-Warshall's algorithm is a recursive algorithm which updates distance matrix $d$ and 
node sequence matrix $s$ in each iteration until it reaches the defined base case. We will run a loop from the first node to the last note. For each iteration, the algorithm simply tells whether a city $k$ needs to be included for shortest in the path between city $i$ is city $j$. It will be updated in node sequence matrix $s$ as part of the shortest path and will be permanently as part of the shortest path. The algorithm behaves like a greedy algorithm as it prefers more nodes and shortest path. We need the latest status of distance matrix $d$ since it stores the current shortest distance between city $i$ is city $j$. The matrix will only be updated when the current shortest distance from is less than the distance if we include a new node $k$ in the shortest path when we compare the distance in each iteration. There is one small problem with this algorithm that it could not record multiple current shortest distance when it happens to be the equality case for the comparision in an iteration. In this case, I choose to include the one with more nodes in the the shortest paths. The implementaion of Floyd-Warshall’s algorithm is shown below:
\begin{lstlisting}[title = $\lstinline{WFI}$]
	for (int k = 1; k <= nodes; k++){
        for (int i = 1; i <= nodes; i++){
            for (int j = 1; j <= nodes; j++){
                //If the path with two edges is less than the path with one edge
                //node that algorithm is from 1 but vector is from 0
                if(i!=k && j!=k && i!=j){ 
                    if (d[i-1][j-1] >= (d[i-1][k-1] + d[k-1][j-1])){
                        //choose the one with more nodes
                        //Set the cost of the edge to be the lesser cost.
                        d[i-1][j-1] = (d[i-1][k-1] + d[k-1][j-1]);
                        //This ensures proper path reconstruction. 
                        //at this point increase to k+1 to continue algorithm
                        s[i-1][j-1] = k;
                    }
                }
            }  	  
		}
    }        
\end{lstlisting}
Reconstructing the shortest path is like going in a zig-zag in the latest node sequence matrix $s$. The element $s[i][j]$ will tell if the node $k$ is needed to be included in the shortest path. If there is an intermidiate $k$ node, the recuresion will be done for $s[i][k]$ and $s[k][j]$ and it will stops when $k=0$ which is the base case. The function $\lstinline{path_recon_to_file}$ is used for path reconstruction and it is shown as below:
\begin{lstlisting}[title = $\lstinline{path_recon_to_file}$]
void path_recon_to_file(std::ofstream& fileOut1, int& n1,int& n2, 
        std::vector<std::vector<int>>& s){
    int k;
    k = s[n1-1][n2-1];//start and destination, remember to -1 in the index
    if (k != 0){
        path_recon_to_file(fileOut1,n1,k,s);//recursive
        fileOut1 << " - "<< k;//add - to separate new nodes
        path_recon_to_file(fileOut1,k,n2,s);//recursive
    }
} 
\end{lstlisting}
To record all the shortest distance and the shortest path to a file, now all we need to do is to run a loop so that it covers all the pair of nodes then run $\lstinline{WFI}$ and $\lstinline{path_recon_to_file}$. This is be done as shown in the implemetation below:
\begin{lstlisting}
for (int i = 1; i <= d.size(); i++){
            for (int j = i+1; j <= d[i].size(); j++){
                //i,j is correspond to algorithm so start from 1
                //j will start from i+1 because we move to other cities
                fileOut1 << i;
                path_recon_to_file(fileOut1,i,j,s);//path reconstruction  
                fileOut1 << " - " <<j; 
                fileOut1 << "\t\t" << d[i-1][j-1];// print distance
                fileOut1 << std::endl;
            }           
        }
\end{lstlisting}
\subsection{Question 3.2}
Implement Dijkstra's algorithm.
\subsubsection{Answer.}
Following the hint, we try to create a class $\lstinline{City}$ with variables $\lstinline{int distance}$ to the store best distance, $\lstinline{bool visited}$ to store whether the city is visit or not and $\lstinline{int cCity}$ to store the connected city. The neccessary procedures such as get and set methods are also included and the class structure is shown as below:
\begin{lstlisting}[title = $\lstinline{City}$]
class City{
    int distance;//store best distance
    bool visited;//visit or not
    int cCity;//other city which it connected to
public:
    City(int distance, bool visited, int cCity){
        set_Values(distance,visited,cCity);//initializer
    };
    void set_Values(int& d, bool& v, int &p);//initializer
    //set methods
    void set_distance(int& d);
    void set_visited(bool& v);
    void set_cCity(int& p);
    //get methods
    int get_distance();
    bool get_visited();
    int get_cCity();
};
\end{lstlisting}
The initialization steps to set visited for all cities to false, set distance for all cities to infinite, set distance for the first city to 0 were done through the code snippets below. We also need the adjacency matrix (distance matrix) which can be done in the same way as described in Floyd-Warshall's implemetation in question 3.1.
\begin{lstlisting}
	int infinity = 1234567;//know from the input file
    //set all to infinity
    std::vector<City> city(nodes,City(infinity,false,0)); 
\end{lstlisting}

\begin{lstlisting}
    //set distance for the first city to 0
    int i_distance = 0;
    bool i_visited = true;
    int i_cCity = 1;
    city[s-1].set_distance(i_distance);
    city[s-1].set_visited(i_visited);
    city[s-1].set_cCity(i_cCity);
\end{lstlisting}
The core of Dijkstra's algorithm to find the shortest route includes 4 steps: (1) Find the city with the lowest distance which has not been visited yet; (2) Mark the city as 'visited' ;(3) If the city is the endpoint, stop; (4) Update the connected cities with the distance found. They are implemented in  $\lstinline{Dijkstra}$ fucntion as described below:\newpage
\begin{lstlisting}[title = $\lstinline{Dijkstra}$]
// Find shortest route, remember algorithm is from 1, vector is from 0.
    int k = s;//set k is the source point
    do{
        for (int j = 1; j <= nodes; j++){
            if((city[j-1].get_visited() == false) && (d[k-1][j-1]!= INFINITY)){
                if(city[j-1].get_distance() >= city[k-1].get_distance() 
                        + d[k-1][j-1]){
                    //adjacent city with lowest distance to the source
                    //updated from distance matrix
                    int n_distance = city[k-1].get_distance() + d[k-1][j-1];
                    city[j-1].set_distance(n_distance);//best distance so far
                    city[j-1].set_cCity(k);//update connected city for shortest
                }        
            }           
        }
        int min = infinity;
        for (int i = 1; i <= nodes; i++){           
            //two loop looks the same but cannot combined because city[i] should
            //not updated on-the-fly. the second loop needs the first loop done
            if((city[i-1].get_visited() == false) && 
                    (city[i-1].get_distance() < min)){
                min = city[i-1].get_distance();
                k = i;
            }
        }
        bool n_visited = true;
        city[k-1].set_visited(n_visited);//mark as visited
    }
    while (k != t);//if the city is the endpoint, stop, molding completed! 
    return city[t-1].get_distance();    
\end{lstlisting}
The construction of the actual route consists of 5 steps: (1) Start at the last city, (2) Look which connected cities has the lowest distance, (3) Add that one to the route, (4) Consider the city just added to be the last one, (5) Is the last city the begin city? Done!. They are implemented in  $\lstinline{path_recon_to_file}$ fucntion as described below:
\begin{lstlisting}[title = $\lstinline{path_recon_to_file}$]
void path_recon_to_file(std::ofstream& fileOut1, int& s, int& t, int& nodes, std::vector<City>& city){
    std::vector<int> path(nodes); 
    int l = 0;
    fileOut1 << s << " - ";
    for (int v = t; v != s; v = city[v-1].get_cCity()){//start at the last city
        path[l++] = v;//add to route
    }
    for (int i = l; i > 1; i--){//add until the first city i=1
        fileOut1 << path[i-1] << " - ";
    }       
    fileOut1 << t;
}
\end{lstlisting}
\subsection{Question 3.3}
The real TomTom problem.
\subsubsection{Answer.}
This question is beyond the scope of this report. An attempt using $\lstinline{std::chrono::time_point}$ was made to measure the run-time of Floyd-Warshall’s algorithm and Dijkstra's algorithm. However, with the given input file, the time to list the shortest path for 3, 10, 50 cites are only from $10^{-4}$ to $0.1$ seconds which does not tell much to comment about which algorithm is better using the codes in this report. Based on complexity analysis \cite[]{ray2013}, Dijkstra's algorithm is faster because it has lower order of complexity. The speed of Dijkstra's algorithm can be even faster with the use of $\lstinline{heap}$ structure and precomputed data. Further study from this report is necessary to understand more about shortest path finding algorithm.
\bibliographystyle{jfm}
% Note the spaces between the initials
\bibliography{ppr}
\end{document}
     