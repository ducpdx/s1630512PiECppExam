\documentclass[a4paper,12pt]{article}%twoside
\usepackage{amsfonts, amsmath, amssymb, bm, color, fancyhdr, graphicx, lastpage, listings, natbib, relsize, titling}
\usepackage{hyperref}
\hypersetup{pdftex,colorlinks=true,allcolors=blue}
\usepackage{hypcap}
\usepackage[left=2.5cm,right=2cm]{geometry}
\definecolor{listinggray}{gray}{0.9}
\definecolor{lbcolor}{rgb}{0.9,0.9,0.9}
\usepackage{caption}
\DeclareCaptionFont{black}{\color{black}}
\DeclareCaptionFormat{listing}{\colorbox{\color{listinggray}}{\parbox{0.988\textwidth}{#1#2#3}}}
\captionsetup[lstlisting]{format=listing,labelfont=white,textfont=white}
\lstset{
backgroundcolor=\color{lbcolor},
    tabsize=4,    
%   rulecolor=,
    language=[GNU]C++,
        basicstyle=\scriptsize,
        breakatwhitespace,
        upquote=true,
        aboveskip={1.5\baselineskip},
        columns=fixed,
        showstringspaces=false,
        extendedchars=false,
        breaklines=true,
        prebreak = \raisebox{0ex}[0ex][0ex]{\ensuremath{\hookleftarrow}},
        frame=single,
        numbers=left,
        showtabs=false,
        showspaces=false,
        showstringspaces=false,
        identifierstyle=\ttfamily,
        keywordstyle=\color[rgb]{0,0,1},
        commentstyle=\color[rgb]{0.026,0.112,0.095},
        stringstyle=\color[rgb]{0.627,0.126,0.941},
        numberstyle=\color[rgb]{0.205, 0.142, 0.73},
%      \lstdefinestyle{C++}{language=C++,style=numbers}’.
}
\lstset{
  backgroundcolor=\color{lbcolor},
  breaklines=true,
  %literate={\_}{}{0\discretionary{\_}{}{\_}}%
  tabsize=4,
  language=C++,
  captionpos=t,
  tabsize=3,
  frame=lines,
  numbers=left,
  numberstyle=\tiny,
  numbersep=5pt,
  showstringspaces=false,
  basicstyle=\footnotesize,
%  identifierstyle=\color{magenta},
  keywordstyle=\color[rgb]{0,0,1},
  commentstyle=\color[rgb]{0,0.6,0},
  stringstyle=\color{red}
  }
 
\newcommand\CC{C\nolinebreak[4]\hspace{-.05em}\raisebox{.4ex}{\relsize{-3}{\textbf{++}}}}

%\fancyhead[LE,RO]{\thepage}
\pagestyle{fancy}
%\fancyhead{}% Remove all header contents
\textheight 24cm
\voffset -0.25cm
\footskip 3.5cm
\setlength{\parindent}{0pt}

\makeatletter
%\renewcommand{\headrulewidth}{0pt}% Remove header rule
\renewcommand{\thefootnote}{\roman{footnote}}	

\renewcommand\section{\@startsection{section}{2}{\z@}%
                                     {-3.25ex\@plus -1ex \@minus -.2ex}%
                                     {1.5ex \@plus .2ex}%
                                     {\normalfont\large\bfseries}}% from \large
\renewcommand\subsection{\@startsection{subsection}{2}{\z@}%
                                     {-3.25ex\@plus -1ex \@minus -.2ex}%
                                     {1.5ex \@plus .2ex}%
                                     {\normalfont\bfseries}}
\renewcommand\subsubsection{\@startsection{subsection}{2}{\z@}%
                                     {-3.25ex\@plus -1ex \@minus -.2ex}%
                                     {1.5ex \@plus .2ex}%
                                     {\normalfont\bfseries\itshape}}% from \large                                                                        

% we use \prefix@<level> only if it is defined
\renewcommand{\@seccntformat}[1]{%
  \ifcsname prefix@#1\endcsname
    \csname prefix@#1\endcsname
  \else
    \csname the#1\endcsname\quad
  \fi}
% define \prefix@section
%\newcommand\prefix@section{Section \thesection: }
\newcommand\prefix@section{}

\renewcommand{\@seccntformat}[1]{%
  \ifcsname prefix@#1\endcsname
    \csname prefix@#1\endcsname
  \else
    \csname the#1\endcsname\quad
  \fi}
% define \prefix@section
%\newcommand\prefix@section{Section \thesection: }
\newcommand\prefix@subsection{}

\renewcommand{\@seccntformat}[1]{%
  \ifcsname prefix@#1\endcsname
    \csname prefix@#1\endcsname
  \else
    \csname the#1\endcsname\quad
  \fi}
% define \prefix@section
%\newcommand\prefix@section{Section \thesection: }
\newcommand\prefix@subsubsection{}

\renewcommand{\baselinestretch}{1.15} 

\pretitle{\noindent\Large\bfseries}
\posttitle{\\}
\preauthor{\itshape}
\postauthor{\\}
\pretitle{\begin{center}\Large\bfseries}
\posttitle{\end{center}\\}
\preauthor{\begin{center}\normalsize}
\postauthor{\end{center}\\}
\predate{\begin{center}\itshape}
\postdate{\end{center}\\}
\setlength{\droptitle}{-1.25 in}
\lhead{PiE $\CC$ Final Assignment}
\rhead{Duc Nguyen \quad \thepage\ / \pageref{LastPage}}
\title{PiE $\CC$ Final Assignment}
%\date{\vspace{-5ex}}
\author{Name: Duc Nguyen, \\student number: s1630512\thanks{Email address for 
correspondence: nguyenmanhduc@student.utwente.nl or ngmaduc@gmail.com}}

\begin{document}
\clearpage\maketitle
\thispagestyle{empty}
\pagestyle{fancy}
\newpage
%\maketitle
\section{Exercise 1}
\subsection{Question 1.1}
Write a $\CC$ program to compute the first N prime numbers, where N is given by the user. Use dynamic arrays to store the primes and use this information in the mod test. 
\subsubsection{Answer.}
Three functions are used for this question including $\lstinline{bruteForce, modTest}$ and 
$\lstinline{print_primes}$ for finding prime numbers and printing out the result in console. Passing by reference are chosen to avoid unnecessary copy of variables.
\begin{lstlisting}
std::vector <unsigned long int> bruteForce (int &n);
std::vector <unsigned long int> modTest (int &n);
void print_primes (int &n, const std::vector <unsigned long int>& primes);
\end{lstlisting}
The method chosen to create dynamic arrays to store the primes is $\lstinline{std::vector}$. The range $[0, 2147483647]$ of  $\lstinline{unsigned long int}$ fits to the scope of the question. The idea of $\lstinline{bruteForce}$,\linebreak $\lstinline{modTestDiv}$ are shown in the following code snippets.
\begin{lstlisting}[title = $\lstinline{bruteForce}$]
       primes.push_back(2);
        unsigned long int c;//need to be the type of primes for mode test
        int count = 1;
        for (int count = 1; count < n; ){//counter from 1; "2" included before
            for (c = 2; c < num; c++){
                if (num % c == 0) {//mod test from 2 to n
                    break;	
                }    
            }
            if (c == num) {//to this point means no divisor up to n, Prime!
                primes.push_back(num);//push to result vector of Prime
                count++;//increase counter   
            }
            num++;  
        }
\end{lstlisting}

\begin{lstlisting}[title = $\lstinline{modTestDiv}$]
        primes.push_back(2);  
        for (int count = 1; count < n; ) {//counter from 1; "2" included before 
            bool isPrime = true;
            for (int i = 0; i < primes.size(); i++){
                if (num % primes[i] ==0) {//non-primes  are products of primes
                  isPrime = false;
                  break;
                }    
            }
            if (isPrime == true) {
                primes.push_back(num);
                count++;//increase counter
            }
            num++;
        }
\end{lstlisting}
\subsection{Question 1.2}
Write to the screen a list of the first 10000 primes in the format below; where $p(n)$ is the
$n^{th}$ prime number. Report only the last five lines. Comment on the behaviour of the ratio $n * ln(p(n))/p(n)$ as n gets large.
\subsubsection{Answer.}
The $\lstinline{void print_ratio (int &n, const std::vector <unsigned long int>& primes)}$ and  prime number finding functions together generate the required ratio. As $n$ gets large, the ratio tends to converge to 1. Until $10^5-th$ prime number, the ratio is 1.103.\\\\
The last five lines are and the $\lstinline{print_ratio}$ are listed below:
\begin{lstlisting}
9996     :       104707  :       1.10348856177824
9997     :       104711  :       1.10356044403989
9998     :       104717  :       1.10361306655082
9999     :       104723  :       1.10366568381267
10000    :       104729  :       1.10371829582629
\end{lstlisting}

\begin{lstlisting}[title = $\lstinline{print_ratio}$]
void print_ratio (int &n, const std::vector <unsigned long int>& primes){
    std::cout << "n\t:\t p(n)\t:\t n*ln( p(n) )/p(n) " << std::endl;
    for (int i = primes.size; i < primes.size(); i++) {
        std::cout << i+1 << "\t:\t" << primes[i] <<"\t:\t" 
            <<std::fixed << std::setprecision(14) << 
                double((i+1)*log(primes[i])/primes[i])<< std::endl;
        }//set precision used for increase decimal displayed
}
\end{lstlisting}
\subsection{Question 1.3}
Based on question 2, give an estimate of the $10^6-th$ prime number.
\subsubsection{Answer.}
We use 1.1 for the value of the ratio with $n=10^6$:
$$10^6 * ln(p(10^6))/p(10^6) \approx 1.1 $$
Using Wolfram Alpha to solve this equation, the estimate of the $10^6-th$ prime number is:
$$p(10^6) \approx 15022800$$
\subsection{Question 1.4}
Instead of writing to the screen, write to a file (on disk) a list containing just the prime
numbers. Print eight numbers per line, such that all numbers have the same space.
\subsubsection{Answer.}

\section{Question 2}
Use symbolic dierentiation from the matlab symbolic toolbox to derive the magnitude of
the force as a function of $r_{ij}$ .\\\\
\textit{ANSWER}
\section{Question 3}
Plot the force fij as a function of the distance $r_{ij}$.\\\\
\textit{ANSWER}
\section{Question 4}
Write two matlab functions that solve the differential equations, for a 1D system of particles, using Euler Forward algorithm and the Verlet algorithm. Comment on which algorithm you prefer and state why?\\\\
\textit{ANSWER}
\section{Question 5}
Plot the potential $V_{ij}$ as a function of the distance rij . Plot separately both the attractive and repulsive contribution.\\\\
\textit{ANSWER}
\section{Question 6}
Plot the potential $V_{ij}$ as a function of the distance rij . Plot separately both the attractive and repulsive contribution.\\\\
\textit{ANSWER}

\end{document}
     